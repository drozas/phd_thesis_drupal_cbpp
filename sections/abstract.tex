\section*{Abstract}

Commons-Based Peer Production (CBPP) is a new model of socio-economic production in which groups of individuals cooperate with each other without a traditional hierarchical organisation to produce common and public goods, such as Wikipedia or GNU/Linux. There is a need to understand how these communities govern and organise themselves as they grow in size and complexity. Following an ethnographic approach, this thesis explores the emergence of and changes in the organisational structures and processes of Drupal: a large and global CBBP community which, over the past fifteen years, has coordinated the work of hundreds of thousands of participants to develop a technology which currently powers more than 2\% of websites worldwide.

Firstly, this thesis questions and studies the notion of contribution in CBPP communities, arguing that contribution should be understood as a set of meanings which are under constant negotiation between the participants according to their own internal logics of value. Following a constructivist approach, it shows the relevance played by less visible contribution activities such as the organisation of events.

Secondly, this thesis explores the emergence and inner workings of the socio-technical systems which surround contributions related to the development of projects and the organisation of events. Two intertwined organisational dynamics were identified: formalisation in the organisational processes and decentralisation in decision-making.

Finally, this thesis brings together the empirical data from this exploration of socio-technical systems with previous literature on self-organisation and organisation studies, to offer an account of how the organisational changes resulted in the emergence of a \textit{polycentric} model of governance, in which different forms of organisation varying in their degree of \textit{organicity} co-exist and influence each other.