\section*{Acknowledgments}

This thesis signifies the most challenging and enthralling intellectual journey of my life. Or, perhaps, a sum of multiple journeys. A multidisciplinary journey, from Computer Science to Sociology. An epistemological journey, from an engineering-shaped mind towards the use of qualitative methods as a tool of immense power to further our understanding of certain questions. An enlightening journey, from participating in free software communities to realising the complexity that lies behind them, and embarking on a quest to understand this complexity as part of the wider phenomenon of Commons-Based Peer Production.  

This journey would never have been possible without the confidence placed on me by Professor Nigel Gilbert, my main supervisor. I met Nigel five years ago when I attended a talk he gave at the Universidad Complutense about a research project named QScience, partly developed on Drupal, which I found by chance in a free software social news website. I became really interested in the research he was leading and I decided to approach him with some more questions after his talk. Luckily for me, he told me they were looking for a Drupal developer and I decided to apply. Weeks later I moved to the UK to work with him as part of the project. I always dreamt of having the opportunity to study a PhD and I felt the dream was closer after being nearer to an academic environment. A few months later, I asked him whether I could refer to him in a PhD application I was preparing, to which he kindly agreed. Some days later he asked me whether I would be interested in doing the PhD at the University of Surrey under his supervision. I still cannot describe the tremendous happiness and honour I felt, and still feel, in that precise moment. I would like to express my most sincere admiration and gratitude for providing me with the opportunity to work together with a scientist of such standing over the past years. His confidence and advice have constantly empowered me in this journey and challenge of becoming a researcher.

This thesis would be a completely different piece of work without the invaluable advice and countless enlightened conversations with Dr Paul Hodkinson, my co-supervisor. My epistemological journey and my passion for increasing my knowledge of Sociology would have never been possible without his mentorship and help. I would like to express my most sincere gratitude for his support and for his time and interest in discussing and supervising my research. 

This research took place in the Centre for Research and Social Simulation (CRESS), the most multidisciplinary and international working environment I have ever had the chance to be in. The innumerous inspiring conversations I had with my colleagues in CRESS over these years deeply enriched this work. But, most importantly, the truly multidisciplinary and diverse environment of CRESS widened my understanding of Science and helped me improve my critical thinking. I would also like to give a special mention for my colleague Dr David Anzola, whose friendship started the moment I joined CRESS and remained even after he finished his PhD and moved back to Colombia, and which I am sure will remain for the rest of our lives. I would also like to thank the researchers from the \textit{Centro Studi di Etnografia Digitale} in Milano and GRASIA in Madrid, who hosted me as a visiting fellow while conducting this research. The brilliant minds and inspiring conversations I had with the researchers of these groups have made an invaluable contribution to this work.

In addition, I would like to thank all of my colleagues from the P2Pvalue project, whose collaboration and inspiration have made me grow as a researcher. The P2Pvalue project funded three of the four years of my research, but most importantly it provided an invaluable opportunity to become involved in and study the world of the commons with some of the most bright, knowledgeable and passionate minds devoted to further our understanding of this phenomenon. I truly hope to continue to contribute to this research field in my future career and to maintain my collaboration and friendship with all of my P2Pvalue colleagues.

I would also like to sincerely thank the Drupal community. What Drupal represents to this work goes beyond a simple case study. Drupal is a community of passionate women and men to create a technology which provides freedom to their users, and shows us how cooperation can triumph over competition. I would also like to thank more specifically all of the Drupalistas with whom I crossed paths for their help to make this research possible, and a special mention must be made to all the informants and interviewees who participated in this research. I would also like to thank more specifically the local communities of London and Madrid, for welcoming me in and helping me with this research as well as in my path to become a Drupalista.

Special thanks are devoted to all the friends I met after moving to Guildford. They provided me with emotional support and we had great moments together: Ana, Sacha, Marianna, Sandra, Belén, Jaime, Pepe, María, Tone, Stephy and so many others! I really hope our paths keep crossing. I would also like to thank the love and support received by my friends in Madrid. A more concrete mention must be made for my closest friends in Madrid: Raúl (now Dr Serrano!) and Antonio. We found each other more than twenty years ago, we grew up together, and in a way we shaped, and keep shaping, each other intellectually. My passion for aiming to untangle the complexity behind any social phenomenon and my critical thinking would have probably never developed in this way without the countless hours we have spent discussing politics, philosophy, history, sociology or technology in the parks of my hometown, Alcorcón. I would also like to thank Blanca for her encouragement to start this journey in times when I was coping with impostor syndrome, and her constantly present emotional support in spite of the physical distance which separated us once I moved to the UK. I also cannot forget to mention Loba, my wonderful dog, who I adopted more than a year ago, just a day after returning back to Madrid. She was sitting next to me in my home office while I was writing most of this work. I still cannot find the words to describe the love and loyalty received by her every day.

I also thank the love and support received from my family, who I always felt next to me in spite of the distance. I especially thank my parents, to whom I dedicate this thesis. I profoundly missed them every single day during the four years I spent in the UK, and I thank them for their constant encouragement, comprehension and love, as well as for all their work and efforts to raise and educate me to become the person I am, with my attributes and flaws. I love you.

Finally, I would like to thank my life partner Tabi. We found each other on a train from Guildford to London two years ago, while I was re-reading Levy's ``Hackers" for this thesis. That initial conversation on the train led to the discovery of a fascinating, caring and loving mind which I hope to keep exploring for the rest of my life. I do not think I could ever have completed this journey without her encouragement and determination over the past two years, which is well illustrated by  the writing she made on my whiteboard, \textit{¡Sí se puede!}, which I could never erase until concluding this thesis. She has been the closest person to accompany me during this amusing journey: the countless hours she spent proofreading this work, the numerous stimulating conversations about the research, and her constant care, love and support cannot be properly thanked with simple words. Neither can I find words descriptive enough to capture the impact she had on my life the moment we found each other. Thus, I can only thank her by expressing my feelings towards her: \textit{te quiero, dormilona}.