\chapter{Interview guide (stage 2)}
\label{appendix-interview-guide-stage-2}

\newpage
	
\hrulefill\par

\begin{center}
\textbf{Interview guide (English)}
\end{center}

\begin{enumerate}
	\item Explain purpose of the research, summarise consent form guidelines and ask for doubts.
	\item Introductory questions:
		\begin{enumerate}
			\item Could you tell me about when and why your interest in Drupal began?
			\item How do you make your living? What is your main role in the projects in which you use Drupal?
			\item Are\slash were you involved in other FLOSS communities? Could you compare your personal experiences of the Drupal community and the others? 
			\item What does Drupal mean to you?
		\end{enumerate}
	
	\item Organisational processes in development of core modules (if relevant for interviewee, skip otherwise):
		\begin{enumerate}
			\item General processes:
			\begin{enumerate}
				\item Have you been involved in the development of core modules? When did you start?
				\item How can people participate in the development of core modules? Have there been more or less people participating over time?
				\item Could you explain to me the day to day in the development of core modules?
				\item How are the decisions taken? For example, adding or not new patches, creating or not a new release, or creating or not a new feature.
			\end{enumerate}	
			\item Concrete process (core initiative leading):
			\begin{enumerate}		
				\item Could you explain to me what is involved by being an initiative leader [if you are/know one]? How were they/you appointed? 
				\item Could you explain to me when the idea of having official initiative leaders started? Why did this happen?
				\item Do you have the impression that the processes related to the development of core are getting more standardised and/or formalised over time?	
			\end{enumerate}		
			\item Reflection in artefacts:
			\begin{enumerate}		
				\item Have there been any changes on Drupal.org regarding this?
			\end{enumerate}
		\end{enumerate}
	
	\item Organisational processes in development of contributed modules (if relevant for interviewee, skip otherwise):
		\begin{enumerate}
			\item General processes:
			\begin{enumerate}
				\item Have you been involved in the development of contributed modules? When did you start?
				\item How is the day to day of maintaining a contributed module?
				\item How can people participate in the development of contributed modules?
			\end{enumerate}

			\item Concrete processes (acceptance/rejection of modules and/or patches \& gaining commit permissions):
			\begin{enumerate}			
				\item How did the possibility of sharing contributed modules ``officially" in Drupal.org started? How does the procedure of accepting or rejecting new modules work? How did it used to work? Have there been any significant changes? Have there been more people involved in taking these decisions over time?
				\item When did you gain commit permissions in Drupal.org? Could you describe what the process was like and what was your experience with it?
				\item Has this process changed over time? Has it become more formalised?
				\item Do you think it has become more difficult to gain permissions over the years?
				\item How are the decisions related to accepting new patches or creating new releases of a contributed module made? Have you applied to become a maintainer of another module, or given permission to another Drupalista to become a maintainer? What was the process like?
			\end{enumerate}

			\item Reflection in artefacts:
			\begin{enumerate}		
				\item Have there been any changes in Drupal.org that made you change the way you organise? For example, changes in the issue list (e.g. new statuses).
			\end{enumerate}

		\end{enumerate}
	
	\item Organisational processes in DrupalCons (if relevant for interviewee, skip otherwise):
		\begin{enumerate}
			\item General processes:
			\begin{enumerate}
				\item Could you describe what a DrupalCon is? Could you explain to me how DrupalCons have been changing over time?	
				\item Have you been involved in the organisation of a DrupalCon? How did you start getting involved in it?
				\item How can people participate in the organisation of DrupalCons? How many people are typically involved in the organisation of a DrupalCon? Have there been more or less people involved in its organisation over time?
			\end{enumerate}
			
			\item Concrete process (selection of presentations):
			\begin{enumerate}			
				\item Have you been involved in the selection of presentations for any DrupalCon? How were you chosen for that?
				\item Could you explain to me how the process of selection of presentations works? Have there been more or less people involved in these processes over time?
				\item Do you think the process is becoming more transparent over time? Is it becoming more formalised? For example, by publishing who the track chairs are on the website.
			\end{enumerate}	

			\item Reflection in artefacts:
			\begin{enumerate}				
			\item Could you indicate if there have been any other changes in the DrupalCon websites regarding the process of selection of presentations? For example, providing specific feedback via the website to the submitters.
			\end{enumerate}
		\end{enumerate}
		
		
	\item Organisational processes in DrupalCamps (if relevant for interviewee, skip otherwise):
		\begin{enumerate}

			\item General processes:
			\begin{enumerate}		
				\item Could you describe what a DrupalCamp is? Have you been involved in the organisation of DrupalCamps? How did you start getting involved in it?
				\item How can people participate in the organisation of DrupalCamp X\footnote{X refers to the specific name of the DrupalCamp(s) in which the interviewee has been involved. For example, DrupalCamp London, DrupalCamp Spain or DrupalCamp North.}? How long has DrupalCamp X been celebrated? How many people are typically involved in the organisation of DrupalCamp X? Have there been more or less people involved in its organisation over time?
				\item Have there been any significant changes in the organisation of DrupalCamp X over the years?
			\end{enumerate}			
	
			\item Concrete process (selection of presentations):
			\begin{enumerate}		
			\item Have you been involved in the selection of presentations for DrupalCamp X?  Could you explain how the process of selection of presentations works? 
			\item How are people responsible for the selection of presentations chosen? Have there been more people involved in it over time?
			\item Do you think the process is becoming more transparent? Is it becoming more formalised? For example, publishing the guidelines for selection criteria or who the track chairs are.
			\end{enumerate}	

			\item Reflection in artefacts:
			\begin{enumerate}				
				\item Could you indicate if there have been any other changes in the DrupalCamp websites, or in any other tools regarding the process of selection of presentations? For example, providing specific feedback via the website or other tools to the submitters.
			\end{enumerate}			
		\end{enumerate}
		
		
	\item Organisational processes in local events (if relevant for interviewee):
		\begin{enumerate}
			\item General processes:
			\begin{enumerate}		
				\item Have you been involved in the organisation of local events? What type of events? Do any of them include presentations?
				\item Could you describe what a [Drupal Local Event]\footnote{This placeholder will be used to indicate the local event or events including presentations in which the interviewee is involved. For example, Drupal Show and Tell.} is? Have you been involved in the organisation of [Drupal Local Event]? How did you start getting involved in it?
				\item How can people participate in the organisation of [Drupal Local Event]? How long has [Drupal Local Event] been celebrated? How many people are typically involved in the organisation of [Drupal Local Event]? Have there been more or less people involved in its organisation over time?
				\item Have there been any significant changes in the organisation of [Drupal Local Event] over the years?
			\end{enumerate}

			\item Concrete process (selection of presentations):
			\begin{enumerate}			
				\item Have you been involved in the selection of presentations for [Drupal Local Event]?  Could you explain to me how the process of selection of presentations works? 
				\item How are the people responsible for the selection of presentations chosen? Have there been more people involved in it over time?
				\item Do you think the process might become more formalised over time? Why/why not?
			\end{enumerate}	
		\end{enumerate}	
	\item Are there any issues that you expected me to ask about, or that you think I should know?
	\item Closing up. Jot down impressions.
\end{enumerate}

\newpage

\hrulefill\par

%%%%%%%%%%%%%%%%%%%%%%%%%%%%%%%%%%%%%%%%%%%% SPANISH %%%%%%%%%%%%%%%%%%%%%%%%%%%%%%%%%%%%%%%%%%

\begin{center}
\textbf{Guión (Español)}
\end{center}

\begin{enumerate}
	\item Explicar el propósito de la investigación, resumir las directrices del formulario de consentimiento y preguntar por posibles dudas respecto al proceso.
	\item Preguntas introductorias:
		\begin{enumerate}
			\item ¿Podrías explicarme cuándo y por qué comenzó tu interés en Drupal?
			\item ¿A qué te dedicas? ¿Cuál es tu función principal en los proyectos en los que has utilizado Drupal?
			\item ¿Participas (o participaste) en otras comunidades de Software Libre? ¿Podrías comparar cómo ha sido tu experiencia personal con la comunidad de Drupal, en comparación con otras comunidades de Software Libre?
			\item ¿Qué significa Drupal para ti?
		\end{enumerate}
	
	\item Procesos organizativos referentes al desarrollo de módulos del núcleo (en caso de que sea relevante respecto a la experiencia del entrevistado/a, omitir en caso contrario):
		\begin{enumerate}
			\item Procesos genéricos:
			\begin{enumerate}
				\item ¿Has participado en el desarrollo de módulos del núcleo? ¿Cuándo empezaste a participar?
				\item ¿Cómo es posible participar en el desarrollo de módulos del núcleo? ¿Ha habido más o menos gente participando a lo largo del tiempo?
				\item ¿Podrías explicarme cómo es el día a día en el desarrollo de módulos del núcleo?
				\item ¿Cómo se toman las decisiones? Por ejemplo, cuando se decide añadir o no un nuevo parche, o cuando se decide acerca de crear una nueva versión, o cuando se decide añadir o no una nueva característica.
			\end{enumerate}	
			\item Proceso concreto (liderazgo de iniciativas del núcleo):
			\begin{enumerate}		
				\item ¿Podrías explicarme en que consiste ser líder de iniciativa del núcleo [si lo eres/conoces a alguien que lo sea]? ¿Podrías explicarme cómo son designados/as?

				\item ¿Podrías explicarme cuándo comenzó la iniciativa que incluye la figura de líderes de iniciativas oficialmente? ¿Por qué ocurrió?
				\item ¿Consideras que los procesos relacionados con el desarrollo del núcleo se han ido estándarizando y/o formalizando con el paso del tiempo?
			\end{enumerate}		
			\item Reflejo en artefactos:
			\begin{enumerate}		
				\item ¿Ha habido algún cambio en Drupal.org debido a los cambios en la organización del desarrollo del núcleo? 
			\end{enumerate}
		\end{enumerate}
		
		
	\item Procesos organizativos referentes al desarrollo de módulos contribuídos (en caso de que sea relevante respecto a la experiencia del entrevistado/a, omitir en caso contrario):
		\begin{enumerate}
			\item Procesos genéricos:
			\begin{enumerate}
				\item ¿Has participado en el desarrollo de módulos contribuídos? ¿Cuándo empezaste a participar?
				\item ¿Podrías explicarme cómo es el día a día de mantener un módulo contribuído?
				\item ¿Cómo es posible participar en el desarrollo de módulos contribuídos?
			\end{enumerate}	
			\item Procesos concretos (aceptación/rechazo de módulos y/o parches y obtener permisos de cambio en el código fuente):
			\begin{enumerate}
				\item ¿Cómo comenzó la posibilidad de compartir módulos contribuídos ``oficialmente" en Drupal.org? ¿Cómo funciona el proceso de aceptación o rechazo de nuevos módulos? ¿Cómo solía funcionar? ¿Ha habido cambios importantes? ¿Ha habido más gente involucrada en la toma de decisiones a lo largo del tiempo?
				\item ¿Cuándo obtuviste permisos para realizar cambios en el código fuente de módulos contribuídos en Drupal.org? ¿Podrías describirme cómo funciona el proceso y cómo fue tu experiencia con el mismo?
				\item ¿Ha cambiado este proceso a lo largo del tiempo? ¿Se ha vuelto más formalizado?
				\item ¿Crees que se ha vuelto más complicado obtener permisos [de realización de cambios en el código fuente de módulos contribuídos] a lo largo del tiempo?
				\item ¿Cómo se toman las decisiones acerca de aceptar o rechazar nuevos parches o crear nuevas versiones de un módulo contribuído? ¿Has aplicado alguna vez para convertirte en mantenedor/a de otro módulo, o has dado permisos a otro Drupalista para que se convierta en mantenedor/a? ¿Cómo fue el proceso?
			\end{enumerate}

			\item Reflejo en artefactos:
			\begin{enumerate}		
				\item ¿Ha habido algún cambio en Drupal.org que provocara cambios en la forma en la que os organizáis? Por ejemplo, cambios en la lista de asuntos (ej.: nuevos estados).
			\end{enumerate}
	\end{enumerate}

	\item Procesos organizativos en DrupalCons (en caso de que sea relevante respecto a la experiencia del entrevistado/a, omitir en caso contrario):
		\begin{enumerate}
			\item Procesos generales:
			\begin{enumerate}
				\item ¿Podrías describir qué es una DrupalCon? ¿Podrías explicarme cómo han ido cambiando las DrupalCons a lo largo del tiempo?
				\item ¿Has estado involucrado/a en la organización de alguna DrupalCon? ¿Cómo comenzaste a involucrarte en ello?
				\item ¿Cómo es posible participar en la organización de DrupalCons? ¿Cuánta gente se involucra normalmente en la organización de una DrupalCon? ¿Ha habido más o menos gente involucrada a lo largo del tiempo?
			\end{enumerate}
			
			\item Proceso concreto (selección de presentaciones):
			\begin{enumerate}			
				\item ¿Has estado involucrado/a en la selección de presentaciones en alguna DrupalCon? ¿Cómo fuíste elegido/a para ello?
				\item ¿Podrías explicarme cómo funciona el proceso de selección de presentaciones? ¿Ha habido más o menos gente involucrada en estos procesos a lo largo del tiempo?
				\item ¿Crees que el proceso se está haciendo de manera más transparente a lo largo del tiempo? ¿Se está volviendo más formal? Por ejemplo, publicando quiénes son los/as encargados/as de la selección de las presentaciones.
			\end{enumerate}	

			\item Reflejo en artefactos:
			\begin{enumerate}				
			\item ¿Podrías identificar si ha habido algún otro cambio en las webs de DrupalCons con respecto al proceso de selección de presentaciones? Por ejemplo, la inclusión de comentarios específicos a los potenciales ponentes.
			\end{enumerate}
		\end{enumerate}
		
		
	\item Procesos organizativos en DrupalCamps (en caso de que sea relevante respecto a la experiencia del entrevistado/a, omitir en caso contrario):
		\begin{enumerate}

			\item Procesos generales:
			\begin{enumerate}	
				\item ¿Podrías describir qué es una DrupalCamp? ¿Has estado involucrado/a en la organización de alguna DrupalCamp? ¿Cómo comenzaste a involucrarte en ello?
				\item ¿Cómo es posible participar en la organización de DrupalCamp X\footnote{X se refiere al nombre específico de la o las DrupalCamp(s) en las que el/la entrevistado/a se ha involucrado. Por ejemplo, DrupalCamp London, DrupalCamp Spain o DrupalCamp North.}? ¿Cuánto tiempo lleva celebrándose la DrupalCamp X? ¿Cuánta gente se involucra normalmente en la organización de DrupalCamp X? ¿Ha habido más o menos gente involucrada a lo largo del tiempo?	
				\item ¿Ha habido cambios significativos en la organización de la DrupalCamp X a lo largo de los años?
			\end{enumerate}			
	
			\item Proceso concreto (selección de presentaciones):
			\begin{enumerate}			
				\item ¿Has estado involucrado/a en la selección de presentaciones en la DrupalCamp X? ¿Podrías explicarme cómo funciona el proceso de selección de presentaciones?
				\item ¿Cómo se escoge a los/as responsables de la selección de presentaciones? ¿Ha habido más o menos gente involucrada a lo largo del tiempo?
				\item ¿Crees que el proceso se está haciendo de manera más transparente a lo largo del tiempo? ¿Se está volviendo más formal? Por ejemplo, publicando los criterios de selección o quiénes son los/as encargados/as de la selección de las presentaciones.
			\end{enumerate}		
				
			\item Reflejo en artefactos:
			\begin{enumerate}				
				\item ¿Podrías identificar algún cambio en las webs de las DrupalCamps, o en alguna otra herramienta, respecto al proceso de selección de presentaciones. Por ejemplo, el envío de comentarios a través de la web, u ofreciendo otras herramientas a los potenciales ponentes.
			\end{enumerate}			
		\end{enumerate}
		
		
	\item Procesos organizativos en eventos locales (en caso de que sea relevante respecto a la experiencia del entrevistado/a, omitir en caso contrario):
		\begin{enumerate}
			\item Procesos generales:
			\begin{enumerate}		
				\item ¿Has estado involucrado/a en la organización de eventos locales? ¿Qué tipo de eventos? ¿Incluye alguno de ellos presentaciones?
				\item ¿Podrías describir qué es un [Drupal Local Event]\footnote{Éste indicador de contenido se utilizará para indicar el evento o eventos locales en los que el/la entrevistado/a está involucrado/a. Por ejemplo, Drupal Show and Tell.} ¿Has estado involucrado/a en la organización de [Drupal Local Event]? ¿Cómo comenzaste a involucrarte en ello?
				\item ¿Cómo es posible participar en la organización de [Drupal Local Event]? ¿Cuánto tiempo lleva celebrándose [Drupal Local Event]? ¿Cuánta gente se involucra normalmente en la organización de [Drupal Local Event]? ¿Ha habido más o menos gente involucrada a lo largo del tiempo?	
				\item ¿Ha habido cambios significativos en la organización de [Drupal Local Event] a lo largo de los años?
			\end{enumerate}

			\item Proceso concreto (selección de presentaciones):
			\begin{enumerate}			
				\item ¿Has estado involucrado/a en la selección de presentaciones en [Drupal Local Event]? ¿Podrías explicarme cómo funciona el proceso de selección de presentaciones?
				\item ¿Cómo se escoge a los/as responsables de la selección de presentaciones? ¿Ha habido más o menos gente involucrada a lo largo del tiempo?
				\item ¿Crees que el proceso quizá se vuelva maś formal a lo largo del tiempo? ¿Por qué?/¿Por qué no?
			\end{enumerate}		

		\end{enumerate}	
	\item ¿Hay algún otro tema del que esperabas que habláramos, o que piensas que debería saber?
	\item Cerrar la conversación. Anotar impresiones.
\end{enumerate}