\setcounter{chapter}{-1}
\chapter{Preface}
\chaptermark{Preface}
\label{chapter:preface}

This study is contextualised under the massive technological, social, political and economic changes experienced over the past twenty years, in which the lines between consumers and producers have become more blurred. This phenomenon has been labelled using the term collaborative economy: an umbrella concept which encompasses disparate initiatives which range from Wikipedia, a free encyclopedia written collaboratively and collectively owned; to Uber, a corporation-owned platform connecting drivers and passengers. Over the course of the years in which this study took place (2013-2017), the collaborative economy has been permeating into more and more aspects of our lives and has been receiving significantly increasing interest far beyond academia. The impact of corporation-owned platforms on diminishing labour rights is nowadays, for example, a common debate in mass media \parencite[e.g.][]{bbc-collab:2017:Online, elpais-collab:2017:Online}. The term collaborative economy has become a priority in the agenda of political institutions \parencite[e.g.][]{euc-collab:2017:Online, generalitatcollab2017}, which is seen as a source of new opportunities and innovation, but also as a source of tensions between previous market operators and the new actors. The rise of the collaborative economy is also employed as an indicator of a systemic contradiction leading to a zero marginal cost economy, in which some authors \parencite[e.g.][]{bauwens2014communism, mason2016postcapitalism} envision the practices from some of the initiatives of the collaborative economy as a key driving force for a possible transition towards a communitarian post-capitalist system.

As recently argued by some scholars \parencite{mayo-unicorns:2017:Online}, significantly different models are emerging around collaborative economy. On the one hand, models in which the value is captured by large corporations which control the platform, communities are dis-empowered and kept out of decision-making, and the technologies and knowledge are proprietary and closed. Uber or AirBnB, a platform connecting people willing to rent rooms or whole properties to guests, are perhaps the most well-known examples of initiatives representing these \textit{corporation-based} models of collaborative economy. On the other hand, models which revolve around a \textit{commons-based} collaborative economy, in which the participants of the community own and self-organise the platform, use and develop technologies that respect the rights of their users, and share the knowledge which has been collaboratively built. Wikipedia or GNU/Linux, a Free/Libre Open Source operating system, are perhaps the most well-known examples of the initiatives around these \textit{commons-based} models. 

This thesis presents a study of self-organisation in a collaborative community focussed on the development of a Free/Libre Open Source Software, named Drupal, whose model responds to the latter: a \textit{Commons-Based} Peer Production community. Drupal is a content management framework, a software to develop web applications, which currently powers more than 2\% of websites worldwide. Since the source code, the computer instructions, was released under a license which allow its use, copy, study and modification by anyone in 2001, the Drupal project has attracted the attention of hundreds of thousands of participants. More than 1.3 million people are registered on Drupal.org, the main platform of collaboration, and communitarian events are held every week all around the World. Thus, as the main slogan of the Drupal project reflects --- ``come for the software, stay for the community", this collaborative project cannot be understood without exploring its community, which is the main focus of this thesis.

In sum, over the course of the next eleven chapters, this thesis presents the story of how hundreds of thousands of participants in a large and global Commons-Based Peer Production community have organised themselves, in what started as a small and amateur project in 2001. This is with the aim of furthering our understanding of how, coping with diverse challenges, Commons-Based Peer Production communities govern and scale up their self-organisational processes.

Chapter \ref{chapter:introduction} provides an overview of the phenomenon of Free/Libre Open Source Software and connects it with that of Commons-Based Peer Production, allowing the theoretical pillars from previous studies on both phenomena to be drawn on. Chapter \ref{chapter:case-study} provides an overview of the main case study, the Drupal community. Throughout the second chapter the Drupal community is framed as an extreme case study of Commons-Based Peer Production on the basis of its growth, therefore offering an opportunity to improve our understanding of how self-organisational processes emerge, evolve and scale up over time in Commons-Based Peer Production communities of this type.

Chapter \ref{sec:theoretical-frameworks} provides an overview of Activity Theory and its employment as an analytical tool: a lens which supports the analysis of the changes experienced in complex organisational activities, such as those from Free\slash Libre Open Source Software communities as part of the wider phenomenon of Commons-Based Peer Production. Subsequently, chapter \ref{chapter:methods}, explores the fundamental methodological aspects considered for this study, which draws on an ethnographic approach. The decision for this approach is reasoned on the basis of the nature of the research questions tackled in the study. Firstly, on requiring an inductive approach, which entails the assumption that topics emerge from the process of data analysis rather than vice versa. Secondly, on the necessity of drawing on a methodological approach which acknowledges the need to understand these topics from within the community.

Chapter \ref{identifyng-contribution:chapter} begins the presentation of the findings of this study. Concretely, it presents the findings regarding the study of contribution in the Drupal community, a notion which is fundamental for the choice of the main unit of analysis, contribution activity, in Activity Theory. The results from this study enabled the identification and consideration, throughout the subsequent chapters, not only of activities which are ``officially" understood as contributions, such as those listed in the main collaboration platform, but also of those which have remained less visible in Free/Libre Open Source software and Commons-Based Peer Production communities and the literature on them.

Having carried out this exploration of the notion of contribution in the Drupal community, the conceptual underpinnings necessary to carry out the study of self-organisation in the Drupal community, by focussing on contribution activities, are laid out. More precisely, chapters \ref{sec:custom-to-contrib} and \ref{chapter:core-system} address the study of the development of projects, activities whose main actions and operations are mostly performed through an online medium;  while chapters \ref{mostly-offline-local:chapter} and \ref{mostly-offline-cons:chapter}, the organisation of events, whose main actions and operations are mostly performed through an offline medium. Throughout these chapters the main argument that binds this thesis together is presented: the growth experienced by the Drupal community led to a formalisation of self-organisational processes in response to a general dynamic of decentralisation of decision-making in order for these processes to scale up. This research identified these two general organisational dynamics, formalisation and decentralisation of decision-making, affecting large and global Commons-Based Peer Production communities as they grow over time. Thus, throughout these chapters, the means by which these general dynamics of formalisation and decentralisation shaped the overall systems which emerged around these different contribution activities are explored. The exploration of the organisational processes of this case study does not only show the existence of these dynamics, but it provides an in-depth account of how these dynamics relate to each other, as well as how they shaped the overall resulting system of peer production, despite the main medium of the peer production activities studied being online/offline, or the significant differences with regard to their main focus of action --- writing source code or organising events. For each pair of chapters this exploration starts with the most informal systems and progresses towards the most formal respectively: \textit{custom}, \textit{contributed} and \textit{core} projects, in chapters \ref{sec:custom-to-contrib} and \ref{chapter:core-system}; and local events, \textit{DrupalCamps} and \textit{DrupalCons}, in chapters \ref{mostly-offline-local:chapter} and \ref{mostly-offline-cons:chapter}.

After carrying out this in-depth exploration of self-organisation, the overall identified changes experienced in the self-organisational processes of the Drupal community are brought together according to general theories of self-organising communities, organisational theory and empirical studies on Commons-Based Peer Production communities, in order to connect the exploration with macro organisational aspects in chapter \ref{multilevel:chapter}. This chapter argues that this study provides evidence of the emergence of \textit{polycentric} governance, in which the participants of this community establish a constant process of negotiation to distribute authority and power over several centres of governance with effective coordination between them. In addition, this chapter argues that the exploration carried out throughout the previous chapters provides an in-depth account of the emergence of an organisational system for peer production in which different forms of organisation, varying in their degree of \textit{organicity}, simultaneously co-exist and interact with each other. Finally, chapter \ref{conclusion:chapter} summarises the main contributions of this thesis and provides a set of implications for practitioners of Commons-Based Peer Production communities.